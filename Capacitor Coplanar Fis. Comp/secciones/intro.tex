\subsection{State of Art.}


Un condensador coplanar es un componente electrónico conformado por dos electrodos ubicados en el mismo plano cubiertos por un medio dieléctrico, es importante destacar que la capacitancia de este arreglo tiene atribuciones derivadas de
un campo eléctrico uniforme entre el grosor de los electrodos y otro no uniforme que se dispersa alrededor del espacio en los extremos de los electrodos, conocido como efecto fringe \ut{[bao]} \ut{[eren]}. Cundo un material externo atraviesa este 
ultimo campo eléctrico, la capacitancia asociada es modificada, por lo que esta configuración de electrodos es util para caracterizar cambios en constantes dieléctricas, por el cambio del material o sus propiedades, alguna de los ejemplos 
donde esta aplicación es util es en las técnicas de evaluación no destructivas, la medición de humedad relativa en suelos, y en la detección de fugas de humedad en edificaciones \ut{[mwelango], [cita], [cita]}.  

Abdollahi-Mamoudan et al.\ut{[cita]} muestra que los factores determinantes en la obtención de un valor de capacitancia optimo son la geometría, el espacio entre placas y la frecuencia, cabe destacar que mientras mas área de un electrodos tenga 
frontera con la otra placa, y la separación entre estas sea mas chica, la capacitancia aumenta. Una forma de poder acreditar las condiciones anteriores es mediante el uso de curvas de llenado del espacio [cita] de grosor controlado 
que separen los electrodos. Debido a que la capacitancia se obtiene mediante el desarrollo de la ecuación de Laplace o Poisson, la geometría vuelve a jugar un papel fundamental en la obtención de la capacitancia de forma analítica. Algunas 
geometrías triviales como el de dos tiras coplanares han sido solucionadas mediante el uso de integrales de Cauchy \ut{[zypman]}, algunas geometrías más complejas como un arreglo de anillos han sido solucionadas mediante el uso de la transformada
de Hankel \ut{[guo]}, no obstante; Parker, Naghed y Wolff, [\ut{cita}] y Campbell [\ut{cita}] han mostrado que este problema es soluble mediante método de diferencias finitas.

Es por esto que este trabajo se propone la caracterización de la capacitancia de una geometría basada en una pseudo-curva de Hilbert de segundo orden, mediante el método de relajación \ut{[cita]}, modificandoló de tal forma que se utilices diferencias 
finitas centradas de cuarto orden. Esto con la finalidad de obtener resoluciones a la ecuación de Laplace y al campo eléctrico de geometrías complejas, así como realizar una discusión de la viabilidad de este método para la obtención teórica de
condensadores para la medición de humedad relativa en suelos. 



\subsection{Marco Teórico.}


Para obtener la capacitancia, se necesita calcular la carga en el dispositivo. Se sabe que las cargas eléctricas generan campos eléctricos y magnéticos, que cumplen las cuatro ecuaciones de Maxwell, una de ellas es la ley de Gauss.


\begin{equation}
    \nabla \cdot \Vec{E} =\frac{\rho_{total}}{\epsilon_0}
\end{equation}

Además, para el caso electrostático (las cargas no se mueven), se cumple la igualdad  $\Vec{E}=-\nabla\phi$, remplazando el campo eléctrico en la ley de Gauss se obtiene \ref{poisson} 


\begin{equation}
     \label{poisson}
    \nabla^2 \phi =\frac{-\rho_{total}}{\epsilon_0}
\end{equation}
Y en una región del espacio donde no hay carga eléctrica, se llega a la ecuación  que es nombrada como ecuación de Laplace \ref{laplace}

\begin{equation}
\label{laplace}
    \nabla^2 \phi =0
\end{equation}

Se sabe que por el teorema de existencia y unicidad, que la solución a una ecuación diferencial es única, por lo tanto si encontramos una solución que cumpla las condiciones de frontera es la única solución al problema. 

\begin{equation}
\label{partials}
\nabla^2 \phi = \frac{\partial^2 \phi}{\partial y^2} + \frac{\partial^2 \phi}{\partial x^2} 
\end{equation}


